\documentclass[10pt]{article}  
\usepackage{graphicx}
\usepackage{geometry}   %设置页边距的宏包

\usepackage{algpseudocode}
\usepackage{comment}
\usepackage{amsmath, amssymb, amsthm}
\usepackage{enumerate}
\usepackage{enumitem}
\usepackage{framed}
\usepackage{verbatim}
\usepackage{microtype}
\usepackage{kpfonts}
\usepackage{multicol}
\usepackage{amsfonts}
\usepackage{array}
\newcommand{\overbar}[1]{\mkern 1.5mu\overline{\mkern-1.5mu#1\mkern-1.5mu}\mkern 1.5mu}
\newcommand{\Ib}{\mathbf{I}}
\newcommand{\Pb}{\mathbf{P}}
\newcommand{\Qb}{\mathbf{Q}}
\newcommand{\Rb}{\mathbf{R}}
\newcommand{\Nb}{\mathbf{N}}
\newcommand{\Fb}{\mathbf{F}}
\newcommand{\Z}{\mathbf{Z}}
\newcommand{\Lap}{\mathcal{L}}
\newcommand{\Zplus}{\mathbf{Z}^+}
\geometry{left=1cm,right=1cm,top=1cm,bottom=1.5cm}  %设置 上、左、下、右 页边距

\begin{document}  
	\begin{multicols}{2}
		\begin{enumerate}
			\item Moment Generating Function\\
			$M_X(t) = E(e^{tx})$\\
			Properties:
			\begin{enumerate}
				\item $M_x(t) = (1-p+pe^t)^n$ for binomial(n,p)
				\item $M_x^{(n)}(0) = E(X^n) \\ \Rightarrow VAR[x]=E[x^2] - E^2[x] = M_x''(0) - [M_x'(0)]^2 $
				\item $M_x(t) = M_y(t) \Rightarrow X Y$ has same distribution
				\item $M_x(0) = 1$
				\item $M_{ax + b}(t) = M_x(at)e^{bt}$
				\item $M_{X+Y}(t) = M_X(t)M_Y(t)$ ($X,Y$ independent)
			\end{enumerate}
			
			\item Gama Function
			$\Gamma(n) = \int_{0}^{\infty}u^{n-1}e^{-u}du = (n-1)!$
			
			\item Basic probability property
			\begin{enumerate}
				\item Conditional Probability and Bayles Theorem\\
				1. $P(A|B) = \frac{P(A\cup B)}{P(B)}$\\
				2. $P(A|B) = \frac{P(B|A)P(A)}{P(B)} = \frac{P(B|A)P(A)}{P(B|A)P(A) + P(B|A')P(A')} $
				\item Expectation\\
				1. $E(X) = \sum xP_X(x)$ or $\int_{-\infty}^{\infty}xf_X(x)$\\
				2. $E(c) = c$\\
				3. $E(aX) = aE(X)$\\
				4. $E(X+Y) = E(X) + E(Y)$\\
				5. $E(X) = M_X'(0)$
				\item Variance and Standard Deviation\\
				1. $VAR(X) = E(X^2)-E^2(X)$ = $\sum[(x-E(X)]P_X(x) = E[(x-\mu)^2] = \sigma^2$\\
				2. $VAR(c) = 0$\\
				3. $VAR(aX) = a^2VAR(x)$\\
				4. $VAR(X\pm Y) = VAR(X) + VAR(Y) \pm 2COV(x,y)$, $COV(x,y) = E(XY)-E(X)E(Y)$
				\item z-score\\
				$Z = \frac{x-\mu}{\sigma}$, measures the distance of x from expected value in standard units.
			\end{enumerate}
			
			\item Discrete Distributions
			\begin{enumerate}
				\item Binomial Distribution\\
				Note as $X \sim B(n,p)$\\
				$P_X(x) = {n \choose x} p^x (1-p)^{n-x}$\\
				$M_X(t) = (1-p+pe^t)^n, E(X) = np, VAR(X) = np(1-p)$
				\item Hyper Geometric Distribution\\
				Note as $X \sim H(N,n,r)$, N:total size, n:total pick, r:size of special subgroup\\
				$P_X(x) = \frac{{r \choose x}{N-r \choose n-x}}{{N \choose n}}$\\
				$E(X) = n\frac{r}{N}, VAR(X) = n\frac{r}{N}(1-\frac{r}{N})(\frac{N-n}{N-1})$
				\item Poisson Process\\
				$P_X(x) = \frac{e^{-\lambda}\lambda^x}{x!}, E(X) = \lambda, VAR(X) = \lambda$\\
				$\lambda$: average arrival in given time or space\\
				When $\lambda = np < 10$, poisson approximates Binomial.
				\item Geometric Distribution
				\begin{enumerate}
					\item X is the r.v of number of total trials (x includes the first success)\\
					$P_X(x) = (1-p)^{x-1}p, x = 1,2,3 \dots$\\
					$E(X) = 1/p, VAR(X) = \frac{1-p}{p^2}$
					\item X is the r.v of number of failed trials (x excludes the first success)\\
					$P_X(x) = (1-p)^xp, x = 0,1,2,\dots$\\
					$M_X(t) = \frac{p}{1-e^t(1-p)}, E(X) = \frac{1-p}{p}, VAR(X) = \frac{1-p}{p^2}$
				\end{enumerate}
				\item Negative Binomial Distribution\\
				X is the r.v of number of trials need to observe the $r^{th}$ success in a sequence of Bernoulli trails where p is the success probability.\\
				$P_X(x) = {x-1 \choose r-1}p^r(1-p)^{x-r}, x = r,r+1,r+2 \dots$\\
				$M_X(t) = (\frac{p}{1-e^t(1-p)})^r, E(X) = \frac{r}{p}, VAR(X) = \frac{r(1-p)}{p^2}$\\
				Alternatively, X is the r.v of failures before the $r^{th}$ success:\\
				$P_X(x) = {x+r-1 \choose r-1}p^r(1-p)^x, x = 0,1,2 \dots$\\
				$M_X(t) = (\frac{1-p}{1-pe^t})^r, E(X) = \frac{r(1-p)}{p}, VAR(X) = \frac{r(1-p)}{p^2}$
			\end{enumerate}
			\item Chebychev's Theorem\\
			$P(\mu-k\sigma \le x \le \mu+k\sigma) \ge 1-\frac{1}{k^2}$
		
			\item Continuous Distributions
			\begin{enumerate}
				\item Uniform(rectangle) Distribution\\
				$f_X(x) = \frac{1}{b-a},  \  (a \le x \le b); F_X(x) = \frac{x-a}{b-a}, a\le x \le b$\\
				$E(X) = \frac{a+b}{2}, VAR(X) = \frac{(b-a)^2}{12}$
				\item Exponential Distribution\\
				$f_X(x) = \frac{1}{\theta}e^{-\frac{x}{\theta}}, (x \ge 0); F_X(x) = 1- e^{-\frac{x}{\theta}}, x \ge 0$\\
				$M_X(t) = \frac{1}{1-\theta t}, E(X) = \theta, VAR(X) = \theta^2$\\
				Note as $X \sim exp(\theta)$, In terms of $\lambda$: $\lambda = \frac{1}{\theta}$
				\item Gamma Distribution\\
				$f_X(x)= \frac{1}{\Gamma(n)\theta^n}x^{n-1}e^{-\frac{x}{\theta}}, x \ge 0$\\
				$M_X(t) = \frac{1}{(1-\theta t)^n}, E(X) = n\theta, VAR(X) = n\theta^2$\\
				In terms of $\alpha, \beta:$ $\alpha = n, \beta = \frac{1}{\theta}$
			\end{enumerate}
			
			\item Central Limit Theorem\\
			Let $\{ X_1, X_2, \dots, X_n \}$ be the independent random variable all of which have the same distribution and mean $\mu$ and standard deviation $\sigma$. If n is large $n \ge 30$, then 
			$ S = X_1 + X_2 + \dots + X_n$
			will be approximately normal with mean $n\mu$ and variance $n\sigma^2$. $(S\sim N(n\mu, n\sigma^2))$\\
			$\textbf{OR} \ S' = \frac{X_1 + X_2 + \dots + X_n}{n}$ will be approximately normal with mean $\mu$, variance $\frac{\sigma^2}{n}$.$(S\sim N(\mu, \frac{\sigma^2}{n}))$
			\item Finding CDF for $Y = g(X)$\\
			$g(X)$ is strictly increasing or decreasing, $f_Y(y) = \frac{d}{dx}F_X(h(y)) \cdot |h'(y)|$		\end{enumerate}
		\newpage
	\end{multicols}
\end{document}